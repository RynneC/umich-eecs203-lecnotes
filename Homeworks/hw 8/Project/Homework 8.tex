\documentclass[12pt]{exam}

% essential packages
\usepackage{fullpage} % margin formatting
\usepackage{enumitem} % configure enumerate and itemize
\usepackage{amsmath, amsfonts, amssymb, mathtools} % math symbols
\usepackage{xcolor, colortbl} % colors, including in tables
\usepackage{makecell} % thicker \Xhline in table
\usepackage{graphicx} % images, resizing

% sometimes needed packages
\usepackage{hyperref} % hyperlinks
% \hypersetup{colorlinks=true, urlcolor=blue}
% \usepackage{logicproof} % natural deduction
% \usepackage{tikz} % drawing graphs
% \usetikzlibrary{positioning}
% \usepackage{multicol}
% \usepackage{algpseudocode} % pseudocode

% paragraph formatting
\setlength{\parskip}{6pt}
\setlength{\parindent}{0cm}

% newline after Solution:
\renewcommand{\solutiontitle}{\noindent\textbf{Solution:}\par\noindent}

% less space before itemize/enumerate
\setlist{topsep=0pt}

% creates \filcl to grey out cells for groupwork grading
\newcommand{\filcl}{\cellcolor{gray!25}}

% creates \probnum to get the problem number
\newcounter{probnumcount}
\setcounter{probnumcount}{1}
\newcommand{\probnum}{\arabic{probnumcount}. \addtocounter{probnumcount}{1}}

% use roman numerals by default
\setlist[enumerate]{label={(\roman*)}}

% creates custom list environments for grading guidelines, question parts
\newlist{guidelines}{itemize}{1}
\setlist[guidelines]{label={}, left=0pt .. \parindent, nosep}
\newlist{gwguidelines}{enumerate}{1}
\setlist[gwguidelines]{label={(\roman*)}, nosep}
\newlist{qparts}{enumerate}{2}
\setlist[qparts]{label={(\alph*)}}
\newlist{qsubparts}{enumerate}{2}
\setlist[qsubparts]{label={(\roman*)}}
\newlist{stmts}{enumerate}{1}
\setlist[stmts]{label={(\roman*)}, nosep}
\newlist{pflist}{itemize}{4}
\setlist[pflist]{label={$\bullet$}, nosep}
\newlist{enumpflist}{enumerate}{4}
\setlist[enumpflist]{label={(\arabic*)}, nosep}

\printanswers

\begin{document}
%%%%%%%%%%%%%%% TITLE PAGE %%%%%%%%%%%%%%%
\title{EECS 203: Discrete Mathematics\\
  Fall 2023\\
  Homework 8}
\date{}
\author{}
\maketitle
\vspace{-50pt}
\begin{center}
  \huge Due \textbf{Thursday, November 2}, 10:00 pm\\
\Large No late homework accepted past midnight.\\
\vspace{10pt}
\large Number of Problems: $7+2$
\hspace{3cm}
Total points: $100+42$
\end{center}
\vspace{25pt}
\begin{itemize}
    \item \textbf{Match your pages!} Your submission time is when you upload the file, so the time you take to match pages doesn't count against you.
    \item Submit this assignment (and any regrade requests later) on Gradescope. 
    \item Justify your answers and show your work (unless a question says otherwise).
    \item By submitting this homework, you agree that you are in compliance with the Engineering Honor Code and the Course Policies for 203, and that you are submitting your own work.
    \item Check the syllabus for full details.
\end{itemize}
\newpage
%%%%%%%%%%%%%%% TITLE PAGE %%%%%%%%%%%%%%% 

\section*{Individual Portion}

\subsection*{\probnum Why You Got a 12-Car Garage? [8 points]}
Ashu recently acquired three 12-car garages, but he has no cars (yet). 
\begin{qparts}
    \item What is the minimum number of cars Ashu has to acquire in order to guarantee that at least one of the garages will have \textbf{more} than 6 cars in it? Justify your answer, including an explanation of why it is the minimum number.
    \item If the garages are all adjacent to one another, what is the minimum number of cars Ashu has to acquire in order to guarantee that the middle garage has more than 6 cars in it? 
\end{qparts}

\begin{solution}
    \begin{qparts}
        \item
            Pigeons: Cars, quantity = $n$\\
            Holes: Garages, quantity = $3$\\
            According to Pigeonhole Principle, we want at least 6 cars in every hole, which means that $\lceil {n\over{3}} \rceil > 6$.\\
            $\therefore n > 18$.\\
            $\therefore$ the minimum number of cars is 19.
        \item
            Consider the worst condition: the left and right garages are all full, while the middle 
            garage only has 6 cars. They add up to $12 \times 2 + 6 = 30$ cars.\\
            Now if we add another car, then it can only go to the middle, so there must be 7 cars after that.\\
            If any one of the left and right garage is not full, then that new car can go to the left or right garage. 
            That makes it possible to keep the middle garage with 6 cars.\\
            $\therefore$ if and only if there are more than 30 cars, it is guaranteed that the middle garage has more 
            than 6 cars in it.\\
            $\therefore$ the minimum number of cars is 31.
    \end{qparts}
\end{solution}

\subsection*{\probnum Sum More Counting [14 points]}
Consider the set of integers between 1 and 18, inclusive. What is the smallest integer $n$ such that, for any subset $S\subseteq \{1, 2, \dots, 18\}$ of size $|S|=n$, there are \textbf{distinct} integers $x, y \in S$ with $x+y=18$? Prove that your answer is sufficient to guarantee this, and the minimum necessary number.

\textbf{Note:} To prove that your choice of $n$ is smallest, you must also give an example of a set of size $|S|=n-1$ that does not contain $x, y \in S$ with $x+y=18$.
\begin{solution}
    The minimum necessary number $n$ is 11.\\
    Consider the partition of the set, shown as elements of a new set:\\
    $\{(1,17),(2,16),(3,15),(4,14),(5,13),(6,12),(7,11),(8,10),9,18\}$\\
    Every element either can form a sum of 18 in its pair or cannot form a sum of 18 with any element (which is 9 and 18).\\
    $\therefore$ There is 10 Holes: partitions of the set, or elements of the new set.\\
    Assume the minimum necessary number is $n$.\\
    Then according to Pigeonhole Principle, $\lceil {n\over 10} \rceil > 1$. (We need at least one full combinition,)\\
    $\therefore n > 10$, the minimum necessary number is 11.\\
    \\Now we show the choice of $11$ is smallest: \\
    Consider the subset $\{9,10,11,12,13,14,15,16,17,18\}$ which has 10 elements, and do 
    not have 2 elements that can sum up to 18

\end{solution}


\subsection*{\probnum Set Sizes [12 points]}

Determine which of these sets are finite, countably infinite, or uncountably infinite. Give a short (about 1 line) explanation for each part.

\begin{qparts}
    \item $\{2, 3\}\times \mathbb{N}$
    \item $(0, 2) - \mathbb{Q}$
    \item $\{x \in \mathbb{R} \mid x^2 - 1 \leq 0\}$
    \item $\{x \in \mathbb{N} \mid x \leq 1000 \}$
\end{qparts}

\begin{solution}
    \begin{qparts}
        \item Countably infinite. For any $x \in \mathbb{N}$, We can map $(2, x)$ to an odd positive integer and $(3,x)$ to an even positive integer, 
        alternately. By doing this we can find a 1-to-1 map from $\{2, 3\}\times \mathbb{N}$ to $\mathbb{Z}^+$. Then we know 
        $|\{2, 3\}\times \mathbb{N}| \leq |\mathbb{Z}^+|$.
        \item Uncountably infinite. Even though there is a set difference from $\mathbb{Q}$, we can still use the table of 
        digit to prove that by diagonalization, that is, to show there does not exist an onto function from $\mathbb{Z}^+$ to $(0, 2) - \mathbb{Q}$. 
        Then $|\mathbb{Z}^+| < |(0, 2) - \mathbb{Q}|$.
        \item Uncountably infinite. $\{x \in \mathbb{R} \mid x^2 - 1 \leq 0\} = [-1,1]$. We can use the same diagonalization method as in (b) to prove that.
        \item Finite. $|\{x \in \mathbb{N} \mid x \leq 1000 \}|$ = 1001. It is just integers from 0 to 1000 inclusively.
    \end{qparts}
\end{solution}

\subsection*{\probnum Ready, set, count! [15 points]}

\textbf{Definition:} $A\oplus B$ is the symmetric difference of the sets $A$ and $B,$ i.e. the set containing all elements which are in $A$ or in $B$ but not in both.

Provide two \textbf{uncountable} sets $A$ and $B$ such that $A\oplus B$ is
\begin{qparts}
    \item finite.
    \item countably infinite.
    \item uncountably infinite.
\end{qparts}
Include in your justification a desciption of the set $A \oplus B$ without reference to the symmetric difference.

\begin{solution}
    \begin{qparts}
        \item $A = \{x \in \mathbb{R} \mid 0 \leq x < 1\} = (0,1]$. \\$B = \{x \in \mathbb{R}\mid 0 \leq x < 1\} = [0,1)$.\\
        $A \oplus B = \{0,1\}$. It is finite, with only two elements 0 and 1.
        \item $A = \mathbb{R} - \mathbb{Q}$. \\$B = \mathbb{R}$. 
        \\These two sets are uncountable, which can be proved by table of digit(diagonalization).\\
        Then $A - B = \mathbb{Q}$, which is countably infinite. This can be proved in the same way we prove $|\mathbb{Z}^+ \times \mathbb{Z}^+|$ is the same size as $|\mathbb{Z}^+|$.
        \item $A = \{x \in \mathbb{R}\mid 0 < x < 1\} = (0,1)$. \\$B = \{x \in \mathbb{R}\mid 1 \leq x < 2\} = [1,2)$. 
        \\Then $A + B = \{x \in \mathbb{R}\mid 0 < x < 2\} = (0,2)$.\\
        We know that they are uncountably infinite in the same logic(diagonalization).
    \end{qparts}
\end{solution}

\subsection*{\probnum Corresponding Counts [18 points]}
Prove that $|[0, 2]| = |(3, 6)|$.

For any functions that you name:
\begin{itemize}
\item Prove that the function is well-defined, i.e.\ that for any $x$ in the domain of your function $f$, $f(x)$ lies in the codomain.
\item Prove any function properties that you use (e.g.\ one-to-one, onto, etc).
\end{itemize}

\begin{solution}
    \begin{qparts}
        \item Proof of $|[0,2]| \leq |(3,6)|$:\\
        We prove it by proving that $\exists$ a one-to-one function $f: [0,2] \rightarrow (3,6)$.
        \\ Consider $f(x) = \frac{x}{2} + 4$, $x \in [0,2]$.\\
        For this function:
        \begin{itemize}
            \item Proof: It is well defined.
            \\It is to prove that: (1) $\forall x \in [0,2], \exists y = f(x), $where $ 3 \leq y \leq 6$.\\
            (2) $\forall y_1, y_2 \in (3,6)$, if 
            $[y_1 \not = y_2] \land [y_1 = f(x_1)] \land [y_2 = f(x_2)]$, then $x_1 \not = x_2$ (One $x$ cannot map to multiple y).
            \\\\
            Proof of (1):\\
            Let $a$ be an arbitrary real numbers in $[0,2]$.\\
            Since $a \geq 0$, $f(a) = \frac{a}{2} + 4 \geq 4 \geq 3$.\\
            Since $a \leq 2$, $f(a) = \frac{a}{2} + 4 \leq 5 \leq 6$.\\
            $\therefore$ $\forall x \in [0,2], \exists y = f(x), $ where $3 \leq y \leq 6$.\\\\
            Proof of (2): \\
            Let $b_1, b_2$ be arbitrary real numbers in $(3,6)$.\\
            Assume $[b_1 \not = b_2] \land [b_1 = f(a_1)] \land [b_2 = f(a_2)]$.
            \\Then $f(a_1) \not = f(a_2)$, $\frac{a_1}{2} + 4 \not = \frac{a_2}{2} +4$, $\frac{a_1}{2} \not = \frac{a_2}{2}$, $a_1 \not = a_2$.\\
            $\therefore \forall y_1, y_2 \in (3,6)$, if 
            $[y_1 \not = y_2] \land [y_1 = f(x_1)] \land [y_2 = f(x_2)]$, then $x_1 \not = x_2$.
            $\therefore$ $f(x)$ is well-defined. 
            \item Proof: It is one-to-one.\\
            It is to prove that $\forall a_1, a_2 \in [0,2]$, $[f(a_1) = f(a_2)]\rightarrow (a_1 = a_2)$.\\
            Let $a_1, a_2$ be arbitrary real numbers in $[0,2]$.\\
            Assume $f(a_1) = f(a_2)$.\\
            Then $\frac{a_1}{2} + 4 = \frac{a_2}{2} + 4$, 
            $\frac{a_1}{2} = \frac{a_2}{2}$, $a_1 = a_2$.\\
            $\therefore f$ is one-to-one.\\
            $\therefore |[0,2]| \leq |(3,6)|$.
        \end{itemize}
        \item Proof of $|(3,6)| \leq |[0,2]|$:\\
        We prove it by proving that $\exists$ a one-to-one function $f: (3,6) \rightarrow [0,2]$.
        \\ Consider $g(x) = \frac{x-3}{3}$, $x \in (3,6)$.\\
        For this function:
        \begin{itemize}
            \item Proof: It is well defined.
            \\It is to prove that: (1) $\forall x \in (3,6), \exists y = g(x), $where $ 0 \leq y \leq 2$.\\
            (2) $\forall y_1, y_2 \in [0,2]$, if 
            $[y_1 \not = y_2] \land [y_1 = g(x_1)] \land [y_2 = g(x_2)]$, then $x_1 \not = x_2$ (One $x$ cannot map to multiple y).
            \\\\
            Proof of (1):\\
            Let $a$ be an arbitrary real numbers in $(3,6)$.\\
            Since $a > 3$, $g(a) = \frac{a - 3}{3} > 0, $ then$ \geq 0$ $(> \rightarrow [> \lor =])$.\\
            Since $a < 6$, $f(a) = \frac{a - 3}{3} < 1 \leq 2$.\\
            $\forall x \in (3,6), \exists y = g(x), $where $ 0 \leq y \leq 2$.\\\\
            Proof of (2): \\
            Let $b_1, b_2$ be arbitrary real numbers in $[0,2]$.\\
            Assume $[b_1 \not = b_2] \land [b_1 = g(a_1)] \land [b_2 = g(a_2)]$.
            \\Then $g(a_1) \not = g(a_2)$, $\frac{a_1 - 3}{3} \not = \frac{a_2 - 3}{3}$, $a_1 -3 \not = a_2 - 3$, $a_1 \not = a_2$.\\
            $\therefore \forall y_1, y_2 \in (3,6)$, if 
            $[y_1 \not = y_2] \land [y_1 = g(x_1)] \land [y_2 = g(x_2)]$, then $x_1 \not = x_2$.
            $\therefore$ $g(x)$ is well-defined. 
            \item Proof: It is one-to-one.\\
            It is to prove that $\forall a_1, a_2 \in (3,6)$, $[gf(a_1) = g(a_2)]\rightarrow (a_1 = a_2)$.\\
            Let $a_1, a_2$ be arbitrary real numbers in $(3,6)$.\\
            Assume $g(a_1) = g(a_2)$.\\
            Then $\frac{a_1 -3}{3}  = \frac{a_2 - 3}{3}$, 
            $a_1 - 3 = a_2 - 3$, $a_1 = a_2$.\\
            $\therefore g$ is one-to-one.\\
            $\therefore |(3,6)| \leq |[0,2]|$.
        \end{itemize}
        \item Due to Schroeder-Bernstein Theorem, since $|[0,2]| \leq |(3,6)|$ and $|(3,6)| \leq |[0,2]|$, $|(3,6)| = |[0,2]|$.
    \end{qparts}
\end{solution}


\subsection*{\probnum Composition Pr$\circ\circ$f [15 points]}

Consider functions $g\colon A \to B$ and $f\colon B \to C.$ \textbf{Prove or disprove} that if $ f $ and $ f \circ g $ are one-to-one, then $ g $ is one-to-one.

\begin{solution}
    Proof of if $ f $ and $ f \circ g $ are one-to-one, then $ g $ is one-to-one:
    \\\\
    Assume $f$ and $f \circ g$ are one-to-one.\\
    According to Definition we know: \\
    (1): $\forall y_1, y_2 \in B$, $[f(y_1) = f(y_2)] \leftrightarrow [y_1 = y_2$].\\
    (2): $\forall x_1, x_2 \in A$, $[f(g(x_1)) = f(g(x_2))] \leftrightarrow [x_1 = x_2$].\\
    We want to show that: $\forall x_1, x_2 \in A, [g(x_1) = g(x_2)] \rightarrow [x_1 = x_2]$.\\\\
    Let $a_1, a_2$ be arbitrary real numbers in A. $b_1 = g(a_1)$, $b_2 = g(a_2)$.\\
    Assume $b_1 = b_2$, that is, $g(a_1) = g(a_2) \in B$.\\
    Then $f(b_1) = f(b_2)$.\\
    Then $f(g(a_1)) = f(g(a_2))$.\\
    Then from (2) we know: $a_1 = a_2$.
    \\$\therefore$ We have proved that if $ f $ and $ f \circ g $ are one-to-one, then $g$ is one-to-one.
    
\end{solution}

\subsection*{\probnum One Hit Wonder [18 points]}
For this problem, we will define two new properties. Let $S$ be a set and $f\colon S \to S$ be some function.

We say $f$ is a \emph{one hit wonder} if:
$$ \forall x \in S\, [ (f \circ f)(x) = f(x) ].$$

Some examples of one-hit wonders from $\mathbb{R} \to \mathbb{R}$ are the absolute value function, 
the ceiling function, and the function which sends every number to 0.

We say $f$ \emph{does nothing} if:
$$ \forall x \in S\, [ f(x) = x ].$$

\begin{qparts}
    \item Prove that if $f$ does nothing, then it is a one-hit wonder.
    \item Prove that if $f$ is a one hit wonder and is one-to-one, then $f$ does nothing.
    \item Prove that if $f$ is a one hit wonder and is onto, then $f$ does nothing.
\end{qparts}


\begin{solution}
    \begin{qparts}
        \item
        Let $a$ be an arbitrary element of $S$.\\
        Assume $f$ does nothing, then $f(a) = a \in S$.
        \\Since $f(a) \in S$, then $\exists f(f(a))$, and $f(f(a)) = f(a) = a$.\\
        Then $f \circ f (a) = f(a)$.\\
        $\therefore$ We have proved that if $f$ does nothing, then it is a one-hit wonder.
    
        \item
        Let $a$ be an arbitrary element of $S$.\\
        Assume $f$ is a one hit wonder and is one-to-one.\\
        Since $f$ is a one hit wonder, $ [(f \circ f)(a) = f(a) ]$, $f(f(a)) = f(a)$.\\
        Since $f$ is one-to-one,$[f(f(a)) = f(a)] \rightarrow f(a) = a$.\\
        $\therefore$ for any $x \in S$, $f(x) = x$, i.e. $f$ does nothing.\\
        $\therefore$ If $f$ is a one hit wonder and is one-to-one, then $f$ does nothing.
    
        \item
        Let $b$ be an arbitrary element of $S$.\\
        Assume $f$ is a one hit wonder and is onto.\\
        Since $f$ is onto, $\exists a \in S,  f(a) = b$.\\
        Since $f$ is a one hit wonder, $f(f(a)) = f(a) = b$.\\
        Since $f(a) = b$ and $f(f(a)) = b$, $f(b)= b$.\\
        $\therefore$ for any $x \in S$, $f(x) = x$, i.e. $f$ does nothing.\\
        $\therefore$ If $f$ is a one hit wonder and is one-to-one, then $f$ does nothing.
    \end{qparts}
\end{solution}


\pagebreak
\setcounter{probnumcount}{1}
\section*{Groupwork}
\subsection*{\probnum Grade Groupwork 7}
Using the solutions and Grading Guidelines, grade your Groupwork 7:
\begin{itemize}
    \item Mark up your past groupwork and submit it with this one.
    \item Write whether your submission achieved each rubric item. If it didn't achieve one, say why not.
    \item Use the table below to calculate scores.
    \item For extra credit, write positive comment(s) about your work.
    \item You don't have to redo problems correctly, but it is recommended!
    \item What if my group changed? \begin{itemize}
        \item If your current group submitted the same groupwork last time, grade it together.
        \item  If not, grade your version, which means submitting this groupwork assignment separately. You may discuss grading together.
    \end{itemize}
\end{itemize}

\begin{center}
\resizebox{\textwidth}{!}{\begin{tabular}{| c | c | c | c | c | c | c | c | c | c | c | c | c |}
\hline
 & (i) & (ii) & (iii) & (iv) & (v) & (vi) & (vii) & (viii) & (ix) & (x) & (xi) & Total:\\
\hline
Problem 2 & & & & &\filcl &\filcl &\filcl &\filcl &\filcl & \filcl& \filcl& \hspace{1cm}/16\\
\hline 
Problem 3 & & &\filcl &\filcl &\filcl &\filcl &\filcl &\filcl &\filcl & \filcl& \filcl& \hspace{1cm}/14\\
\Xhline{1.25pt}
Total: &\filcl &\filcl &\filcl &\filcl &\filcl &\filcl &\filcl &\filcl & \filcl& \filcl& \filcl&\hspace{1cm}/30\\
\hline
\end{tabular}}
\end{center}

\subsection*{Previous Groupwork 7(1): Raise the Roof [16 points]}
Let $f\colon \mathbb R \rightarrow \mathbb Z$, where $f$ is defined as $f(x)= \left \lceil \frac{ x+5 }{2} \right \rceil + 12.$ Is $f$ onto? Prove your answer.

\begin{solution}
    $f$ is onto.\\
    i.e. $\forall b \in \mathbb{Z}$, $\exists a \in \mathbb{R}$ such that $b = f(a)$.
    Proof:\\
    Let $b$ be an arbitrary integer.\\
    Consider $a = 2b -29$.\\
    Then $a$ is also an integer, $a \in \mathbb{Z} \subseteq \mathbb{R}$, is in the domain.\\
    Then
    \begin{align*}
        f(a)  &= \lceil \frac{ 2b - 29+5 }{2}\rceil + 12\\
              &= \lceil \frac{ 2b - 24 }{2} \rceil + 12\\
              &= \lceil b - 12 \rceil + 12
    \end{align*} 
    Since $b$ is an integer, $\lceil b - 12 \rceil = b -12$.\\
    $\therefore f(a) = b - 12 + 12 = b$.\\
    $\therefore \forall b \in \mathbb{Z}$, $\exists a \in \mathbb{R}$ such that $b = f(a)$.\\
    $\therefore f$ is onto.
\end{solution}

\subsection*{Previous Groupwork 7(2):  You Mod Bro? [14 points]}
Find all solutions of the congruence $12x^2 + 25x \equiv 10 \pmod{11}$.
\begin{solution}
    $12x^2 + 25x \equiv 10 \pmod{11}$\\
    $(11+1)x^2 + (2 \cdot 11 +3)x \equiv 10 \pmod{11}$\\
    $x^2 + 3x - 10 \equiv 0 \pmod{11}$\\
    $(x+5)(x-2) \equiv 0 \pmod{11}$ \\
    $\therefore x \equiv 2 \pmod{11}$ or $x \equiv -5 \pmod{11}\equiv 6 \pmod{11}$.\\
    $\therefore$ The solutions are: all integer $x$ satisfying $x \equiv 2 \pmod{11}$ or $x\equiv 6 \pmod{11}$.
\end{solution}

\subsection*{\probnum Divisibility by Seven [12 points]}

In this question we will show that, given a 7-digit number, where all digits except perhaps the last are non-zero, you can cross out some digits at the beginning and at the end such that the remaining number consists of at least one digit and is divisible by 7. You are allowed to cross off zero digits.

For example, if we take the number 1234589, then we can cross out 1 at the beginning and 89 at the end to get the number $2345 = 7 \cdot 335$.

We will label the digits of an arbitrary 7-digit number as $$x_6 x_5 x_4 x_3 x_2 x_1 x_0.$$
\begin{qparts}
    \item Prove that there exists some $i < 7$ such that either $x_i x_{i-1} \dots x_0$ is divisible by 7,
    or, if it isn't, then there exists some $j < i$ such that $x_j x_{j-1} \dots x_0$ is congruent to it
    modulo 7.
    \item Use part (a) to prove that if there does not exist some $i < 7$ such that $x_i x_{i-1} \dots x_0$ is divisible by 7, then there exists $7 > i > j \geq 0$ so that $$\underbrace{x_ix_{i-1}\dots x_{j+1} 0 \dots 0}_{i + 1 \text{ digits total}}$$ is divisible by 7.
    \item Prove the full claim. That is, show that, given a 7-digit number, where all digits except perhaps the last are non-zero, you can cross out some digits at the beginning and at the end such that the remaining number consists of at least one digit and is divisible by 7.
\end{qparts}
\begin{solution} 
    \begin{qparts}
        \item
        There are 7 numbers: $x_6 x_5 x_4 x_3 x_2 x_1 x_0$, $x_5 x_4 x_3 x_2 x_1 x_0$, $\cdots$, $x_0$.\\
        Use them as Pigeons.\\
        The remainer of any number mod 7, if the number cannot be divided by 7, has 6 circumstances: 1,2,3,4,5,6.\\
        Use them as Holes.\\
        Assume that all these 7 numbers is not divisible by 7, then there are 7 Pigeons in 6 holes, which means that there are at least 2 Pigeons in one whole.\\
        This means that $i < 7$ and $j < i$ such that $ = x_i x_{i-1} \dots x_0$ and $x_j x_{j-1} \dots x_0$ have the same remainer when divided by 7.\\
        Mark the two numbers as $a $ and $b$.\\
        Then $b \equiv a \pmod{7}$.\\
        $(b-a) \equiv 0 \pmod{7}$.\\
        Then we have proved the statement: There exists some $i < 7$ such that either $x_i x_{i-1} \dots x_0$ is divisible by 7,
        or, if it isn't, then there exists some $j < i$ such that $x_j x_{j-1} \dots x_0$ is congruent to it
        modulo 7.
        \item
        If there does not exist some $i < 7$ such that $x_i x_{i-1} \dots x_0$ is divisible by 7, then from (a) who know: there exists some $j < i$ such that $x_j x_{j-1} \dots x_0$ is congruent to it.\\
        i.e. $(x_i x_{i-1} \dots x_0 - x_j x_{j-1} \dots x_0) \equiv 0 \pmod{7}$.\\
        that is, $x_i x_{i-1} \cdots x_{j+1} 0 \cdots 0 \equiv 0 \pmod{7}$.\\
        $\therefore$ We have proved the statement. 
        \item 
        Given a 7-digit number $x_6 x_5 x_4 x_3 x_2 x_1 x_0$,\\
        Case 1: $\exists i < 7$ such that $x_i x_{i-1} \dots x_0$ is divisible by 7.\\
        Then if $i = 6$, the whole digit is divisible by 7, else we can just cross out all digits from $x_6$ to $x_{i+1}$. The remaining 
        digit $x_i x_{i-1} \dots x_0$ is divisible by 7.\\\\
        Case 2:
        $\not\exists i < 7$ such that $x_i x_{i-1} \dots x_0$ is divisible 
        by 7.\\
        Then through (b) we know: there exists some $j < i$ such that 
        $x_ix_{i-1}\dots x_{j+1} 0 \dots 0$ is divisible by 7.\\
        We know that $x_ix_{i-1}\dots x_{j+1} 0 \dots 0 = x_ix_{i-1}\dots x_{j+1} \times 100\cdots$\\
        Then $x_ix_{i-1}\dots x_{j+1} \times 100\cdots \equiv 0 \pmod{7}$.\\
        Since $100\cdots \not\equiv 0 \pmod{7}$\\
        There must be $x_ix_{i-1}\dots x_{j+1} \equiv 0 \pmod{7}$.\\
        So we can rule out all digits before $x_i$ and after $x_j$ to get a number divisible by 7.
    \end{qparts}
\end{solution}

\subsection*{\probnum A Powerful Proof [30 points]}

In this question we will prove that for any set $X,$ $|\mathcal{P}(X)| > |X|$ ($\mathcal{P}(X)$ is the power set of $X$). Note that 
while this is simple in the case where $X$ is finite, things get more complicated when we allow $X$ to be infinite. This proof covers all cases.

\begin{qparts}
    \item Show that for all (possibly infinite) sets $X$, $|\mathcal{P}(X)| \ge |X|.$
    \item Let $g\colon X \to P(X)$ be an arbitrary function.
    Show that the set $D := \{a\in X \mid a\notin g(a)\}$ is not in the range of $g$.
    \item Explain why this shows that $|P(X)| \leq |X|$ is false and conclude the proof.    
    \item Based on your conclusions above, are there uncountable sets ``larger" than $\mathbb{R}$? Explain.
\end{qparts}

\begin{solution}
    \begin{qparts}
        \item
        We define $f: X \rightarrow \mathcal{P}(X)$: \\
        The map of every element in $X$ to a set in $\mathcal{P}(X)$ that 
        only contains that element.\\
        That is: for $X = \{a,b,c,\cdots\}$, $f(a) = \{a\}$, $f(b) = \{b\}$, $\cdots$.\\
        Since for every element in $X$, we can find a subset of $X$ that only contains that element, 
        and every subset is unique, we know this is a one-to-one function.\\
        $\therefore |X| \leq |\mathcal{P}(X)|$.
        \item
        The range of $g$ is $\{ g(a) | a \in X\}$. \\
        $D$ is a subset of $X$ which contains all the elements that are not an element of its image.\\
        Assume $D \in range(X)$, then $\exists a \in X$ such that $g(a) = D$, which is an element of $\mathcal{P}$.\\
        Due to the definition of $D$, $D = \{ a \in X | a \not \in g(a)\}$, then $X \in D$. This causes contradiction.
        $\therefore D$ is not in the range of $g$.
        \item
        Since every element in $D$ is also in $X$, it is a subset of $X$, so $D$ is an element of $\mathcal{P}(X)$, which is 
        in the codomain of $g$.\\
        However, $D$ is not in the range of $g$.\\
        $\therefore$ the $range(g) < codom(g)$, there does not exist an onto function $g$ from $X$ to $\mathcal{P}(X)$.\\
        $\therefore$ $|\mathcal{P}(X)| \leq |X|$ is false, $|\mathcal{P}(X)| > |X|$.
        \item
        From c, we can know that $| \mathcal{P}(\mathbb{R}) | > |\mathbb{R}|$. This is a ``larger" set than $\mathbb{R}$.

    \end{qparts}
\end{solution}


\end{document}
