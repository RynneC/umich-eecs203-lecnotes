\documentclass[12pt]{exam}
% essential packages
\usepackage{fullpage} % margin formatting
\usepackage{enumitem} % configure enumerate and itemize
\usepackage{amsmath, amsfonts, amssymb, mathtools} % math symbols
\usepackage{xcolor, colortbl} % colors, including in tables
\usepackage{makecell} % thicker \Xhline in table
\usepackage{graphicx} % images, resizing
% \usepackage{todonotes}
% \newcommand{\gbnote}[1]{\todo[color=blue!20]{Greg's Note: #1}}
% \newcommand{\gbnoteinline}[1]{\todo[inline, size=\normalsize, color=blue!20]
% sometimes needed packages
% \usepackage{hyperref} % hyperlinks
% \hypersetup{colorlinks=true, urlcolor=blue}
% \usepackage{logicproof} % natural deduction
% \usepackage{tikz} % drawing graphs
% \usetikzlibrary{positioning}
% \usepackage{multicol}
% \usepackage{algpseudocode} % pseudocode
% paragraph formatting
\setlength{\parskip}{6pt}
\setlength{\parindent}{0cm}
% newline after Solution:
\renewcommand{\solutiontitle}{\noindent\textbf{Solution:}\par\noindent}
% less space before itemize/enumerate
\setlist{topsep=0pt}
% creates \filcl to grey out cells for groupwork grading
\newcommand{\filcl}{\cellcolor{gray!25}}
% creates \probnum to get the problem number
\newcounter{probnumcount}
\setcounter{probnumcount}{1}
\newcommand{\divides}{\,|\,}
\newcommand{\probnum}{\arabic{probnumcount}. \addtocounter{probnumcount}{1}}
% use roman numerals by default
\setlist[enumerate]{label={(\roman*)}}
% creates custom list environments for grading guidelines, question parts
\newlist{guidelines}{itemize}{1}
\setlist[guidelines]{label={}, left=0pt .. \parindent, nosep}
\newlist{gwguidelines}{enumerate}{1}
\setlist[gwguidelines]{label={(\roman*)}, nosep}
\newlist{qparts}{enumerate}{2}
\setlist[qparts]{label={(\alph*)}}
\newlist{qsubparts}{enumerate}{2}
\setlist[qsubparts]{label={(\roman*)}}
\newlist{stmts}{enumerate}{1}
\setlist[stmts]{label={(\roman*)}, nosep}
\newlist{pflist}{itemize}{4}
\setlist[pflist]{label={$\bullet$}, nosep}
\newlist{enumpflist}{enumerate}{4}
\setlist[enumpflist]{label={(\arabic*)}, nosep}
\printanswers
\begin{document}
%%%%%%%%%%%%%%% TITLE PAGE %%%%%%%%%%%%%%%
\title{EECS 203: Discrete Mathematics\\
Fall 2023\\
Homework 3}
\date{}
\author{}
\maketitle
\vspace{-50pt}
\begin{center}
\huge Due \textbf{Thursday, Sept. 21}, 10:00 pm\\
\Large No late homework accepted past midnight.\\
\vspace{10pt}
\large Number of Problems: $6+2$
\hspace{3cm}
Total Points: $100+30$
\end{center}
\vspace{25pt}
\begin{itemize}
\item \textbf{Match your pages!} Your submission time is when you upload the
file, so the time you take to match pages doesn't count against you.
\item Submit this assignment (and any regrade requests later) on Gradescope.
\item Justify your answers and show your work (unless a question says
otherwise).
\item By submitting this homework, you agree that you are in compliance with
the Engineering Honor Code and the Course Policies for 203, and that you are
submitting your own work.
\item Check the syllabus for full details.
\end{itemize}
\newpage
%%%%%%%%%%%%%%% TITLE PAGE %%%%%%%%%%%%%%%
\section*{Individual Portion}
\subsection*{\probnum Division Tradition [18 points]}
Let the domain of discourse be positive integers. The notation $a\divides b$ means
``$a$ divides $b,$'' or more formally: ``there exists an integer $q$ such that $b =
a\cdot q.$"
Prove the following statements:
\begin{qparts}
\item For all $k, n$, if $k\divides n$, then $k^2\divides n^2$.
\item For all $k, n_1, n_2$, if $k\divides n_1$ and $k\divides n_2$, then $k\
divides (n_1+n_2)$.
\item For all $a, b, c$, if $a\divides b$ and $b\divides c,$ then $a\divides
c$.
\end{qparts}
\begin{solution}

\begin{qparts}
\item 
Let $k$ be an arbitrary integer. \\
Since $k \divides n$, let $n = p \cdot k$, $p$ is an integer.
then $n^2 = p^2 \cdot k^2$. \\
Since $p$ is an integer, $p^2$ is also an integer.
\\ Therefore we have proved that $k^2 \divides n^2$.

\item 
Let $k$ be an arbitrary integer. \\
Since $k \divides n_1$, let $n_1 = p \cdot k$, $p$ is an integer. \\
Since $k \divides n_2$, let $n_2 = q \cdot k$, $q$ is an integer. \\
then $(n_1 + n_2) = (p + q)  \cdot k$. \\
Since $p,q$ are integers, $(p+q)$ is an integer.\\
Therefore we have proved that $k \divides (n_1 + n_2)$.

\item 
Let $a$ be an arbitrary integer. \\
Since $a \divides b$, let $b = p \cdot a$, $p$ is an integer. \\
Since $b \divides c$, let $c = q \cdot b$, $q$ is an integer. \\
then $c = (p \cdot q) \cdot a$. \\
Since $p,q$ are integers, $(p \cdot q)$ is an integer.\\
Therefore we have proved that $a \divides c$.
\end{qparts}

\end{solution}
\subsection*{\probnum Even Stevens Rerun [15 points]}
Let $n$ be an integer. Using only the definitions of even and odd, prove that these
statements are equivalent:
\begin{stmts}
\item $n+1$ is odd
\item $5n-7$ is odd
\item $n^2$ is even
\end{stmts}
\textbf{Hint:} You can prove that these three statements are equivalent in a
circular way by showing $(\text i) \rightarrow (\text{ii}), (\text{ii}) \rightarrow
(\text{iii}),$ and $(\text{iii}) \rightarrow (\text i)$.
\begin{solution}
\begin{qparts}
    \item $(\text i) \rightarrow (\text{ii})$\\
Let $(n+1)$ be an arbitrary odd integer. There exist integer $k$ such that $(n+1) = 2k+1$. \\
Then $5n-7 = 5(n+1) - 12 = 10k-7 = 2(5k -4)+1$. \\
Since $k$ is an integer, $5k-4$ is an integer, $5n-7 = 2(5k-4)+1$ is an odd integer.\\
Therefore we have proved that $(\text i) \rightarrow (\text{ii})$.

    \item $(\text{ii}) \rightarrow (\text{iii})$\\
    Let $(5n-7)$ be an arbitrary odd integer. There exist integer $k$ such that $(5n-7) = 2k+1$.
    \\
Then 
\begin{align*}
(5n-7)^2 & = 25n^2 -70n + 49 = 4k^2 + 4k + 1 \\
25n^2 & = 4k^2 + 4k + 70n -48
 \end{align*}
Since $k$ is an integer, $(2k^2 + 2k + 35 -24)$  is an integer, and since $(4k^2 + 4k + 70n -48) = 2(2k^2+2k+35n-24)$,  $25n^2$ is even.\\
    And since $n^2$ is an integer, for the odd integer 25, $n^2$ must be an even integer to make their product $25n^2$ an even integer. \\
    Therefore we have proved that $(\text{ii}) \rightarrow (\text{iii})$.
    
    \item $(\text{iii}) \rightarrow (\text{i})$\\
To prove “If $n^2$ is even, then $n+1$ is odd", we can prove its contrapositive, that is, “If $n+1$ is even, then $n^2$ is odd."\\
let $n+1$ be an arbitrary even integer, then there exist integer $k$ such that $n + 1 = 2k$.\\
Then $n^2 = (2k-1)^2 = 4k^2 - 4k + 1$.\\
Since $k$ is an integer, $4k^2 - 4k = 2(2k^2-2k) $ is an even integer, and therefore $n^2 = 4k^2 - 4k + 1 = 2(2k^2 -2k)+1$ is an odd integer.\\
Then we have proved $(\text{iii}) \rightarrow (\text{i})$ by proving its contrapositive.\\
    \item 
Therefore, $(\text i) \rightarrow (\text{ii}) \rightarrow (\text{iii}) \rightarrow (\text{i})$...... We can deduce that $(\text{i}) \longleftrightarrow (\text{ii}) \longleftrightarrow (\text{iii})$

\end{qparts} 

\end{solution}
\subsection*{\probnum Even the Odds [15 points]}
\textbf{Prove or disprove:} there exists an integer $n$ where $n$ is even and $n^2
+ 4$ is odd.
\begin{solution}
We can disprove it by proving its negation: ``There does not exists an integer $n$ where $n$ is even and $n^2+ 4$ is odd.'' \\
Let $n$ be an arbitrary odd integer. Then there exist an integer $k$ such that $n = 2k$.\\
Then $n^2 + 4 = (2k)^2 + 4 = 4k^2 + 4 = 2(2k^2+2)$.\\
since $k$ is an integer, $(2k^2+ 2)$ is an integer, therefore $2(2k^2+2)$ must be an odd integer. So when $n$ is even, $n^2 + 4$ can not be odd.

\end{solution}
\subsection*{\probnum To Prove or Not to Prove [18 points]}
\textbf{Prove or disprove} the following statements where the domain of discourse
is all integers:
\begin{qparts}
\item For all $x$ there exists a $y$ such that $x^2+y = 2.$
\item For all $y$ there exists an $x$ such that $2x-y=8.$
\item There exists an $x$ such that for all $y,$ $\frac{y}{x} = y.$
\item There exists a $y$ such that for all $x,$ $x^2+y = 2.$
\end{qparts}
\begin{solution}
\begin{qparts}
\item 
Prove it.\\
Let $x$ be an arbitrary integer. \\
Since every integer has its square that is an integer, $x^2 $ is an integer. \\
Let $y = 2 - x^2$. Then $y$ is an integer \\
Then we have $y$ such that $x^2 + y = 2$.

\item Disprove it.\\
We will prove its negation: "Therefore there exists an $y$ that in the domain of all integers, for all $x$, $2x-y \neq 8$."
For all $y$ there exists an $x$ such that $2x-y=8.$
Consider $y = 1$.\\
If there exists an $x$ such that $2x - y = 8$, then $x = \frac{8+y}{2} = 4.5$, which is not an integer.\\
Therefore there exists an $y$ that in the domain of all integers, for all $x$, $2x-y \neq 8$.
\item 
Prove it.\\
Consider $x = 1$. Let $y$ be an arbitrary integer. \\
Then $\frac{y}{x} = y$. \\
Therefore when $x=1$, for all $y$, $\displaystyle \frac{y}{x} = y$.
\item 
Disprove it. \\
We will prove its negation:  "There does not exist a $y$ such that all $x,$ $x^2+y = 2.$"\\
Seeking a contradiction, assume that for integer $y$,  $x^2+y = 2$ for all $x$.\\
Then when $x = 1$, y should be $2-x^2 = 1$.\\
When $x = 2$, $y$ should be $2-x^2= -2$. \\
This completes the contradiction since $y$ can not be $1$ and $-2$ at the same time.\\
Therefore we have proved that there do not exist a $y$ such that all $x,$ $x^2+y = 2.$
\end{qparts}

\end{solution}
\subsection*{\probnum What's Your Rationale? [20 points]}
Prove that for any rational number and any irrational number, there exists an
irrational number between them.
You may assume the following without proof:
\begin{itemize}
\item The sum of a rational and irrational is irrational.
\item The product of a nonzero rational and irrational is irrational.
\end{itemize}
\begin{solution}
Let $x$ be an arbitrary rational number, and $y$ be an arbitrary irrational number.\\
Without Loss of Generality (due to symmetry), assume $x<y$.\\
We always have $\frac{x+y}{2}$:\\
$\frac{x+y}{2} = \frac{x}{2} + \frac{y}{2} < \frac{y}{2} + \frac{y}{2} = y$. \\
$\frac{x+y}{2} = \frac{x}{2} + \frac{y}{2} > \frac{x}{2} + \frac{x}{2} = x$. \\
Then we know that $x < \frac{x+y}{2} < y$.\\
Since $x$ is rational, $y$ is irrational, and the sum of a rational and irrational is irrational, $\frac{x+y}{2}$ is irrational.\\
Therefore we have proved that for any rational number $x$ and any irrational number $y$, there exists an irrational number between them.

\end{solution}
\subsection*{\probnum $\text{Contra}>0$ [14 points]}
Let $x$ be an integer. Prove that if $6x + 2$ is negative, then $x$ is negative
using
\begin{qparts}
\item a proof by contrapositive.
\item a direct proof.
\end{qparts}
\begin{solution}
\begin{qparts}
\item 
We will prove it by proving its contrapositive: "Let $x$ be an integer. If $x$ is not negative, then $6x+2$ is not negative".\\
Let $x$ be an arbitrary integer in the domain that $x\ge$ 0.\\
Then $6x + 2 \ge 2 > 0$, is not negative.\\
Therefore we have proved the original proposition by proving its contrapositive.\\
\item 
Let $6x + 2$ be an arbitrary integer in the domain that $6x+2 < 0$.\\
Then $6x < -2$, $x < -\frac{1}{3} < 0$.\\
Therefore we have proved that if $6x + 2$ is negative, then $x$ is negative.
\end{qparts}
\end{solution}
\pagebreak
\setcounter{probnumcount}{1}
\section*{Groupwork}
\subsection*{\probnum Grade Groupwork 2}
Using the solutions and Grading Guidelines, grade your Groupwork 2:
\begin{itemize}
\item Mark up your past groupwork and submit it with this one.
\item Write whether your submission achieved each rubric item. If it didn't
achieve one, say why not.
\item Use the table below to calculate scores.
\item For extra credit, write positive comment(s) about your work.
\item You don't have to redo problems correctly, but it is recommended!
\item What if my group changed? \begin{itemize}
\item If your current group submitted the same groupwork last time, grade
it together.
\item If not, grade your version, which means submitting this groupwork
assignment separately. You may discuss grading together.
\end{itemize}
\end{itemize}
\begin{center}
\resizebox{\textwidth}{!}{\begin{tabular}{| c | c | c | c | c | c | c | c | c | c |
c | c | c |}
\hline
& (i) & (ii) & (iii) & (iv) & (v) & (vi) & (vii) & (viii) & (ix) & (x) & (xi) &
Total:\\
\hline
Problem 1 & & & & & & & & &\filcl & \filcl& \filcl& \hspace{1cm}/13\\
\hline
Problem 2 & & & & & & & &\filcl &\filcl & \filcl& \filcl& \hspace{1cm}/12\\
\Xhline{1.25pt}
Total: &\filcl &\filcl &\filcl &\filcl &\filcl &\filcl &\filcl &\filcl & \filcl& \
filcl& \filcl&\hspace{1cm}/25\\
\hline
\end{tabular}}
\end{center}
\subsection*{Previous Group Homework 2(1): Implication Inception [13 points]}
Consider two propositions, $A$ and $B$.

\begin{qparts}
    \item Prove via a truth table that $A \equiv \left[ (A \to B) \to A  \right]$.
    \item Consider the compound proposition $$\underbrace{((A\to B) \to A) \to B \dots}_{203 \text{ letters}}$$ which has $203$ total letters in it, alternating between $A$ and $B$. Fill in the rest of the following truth table. You don't need to manually make all $203$ columns of the truth table to solve this, so try to find a pattern and think of a shortcut.
    \begin{center}
            \begin{tabular}{c c | c}
            $A$ & $B$ & $\underbrace{((A\to B) \to A) \to B \dots}_{203 \text{ letters}}$ \\
            \hline
            $T$ & $T$ & \\
            $T$ & $F$ & \\
            $F$ & $T$ & \\
            $F$ & $F$ & \\
            \end{tabular}
    \end{center}
    \item Consider the following truth table. This is similar to the truth table in part (b), but it contains each iteration of the compound proposition from $1$ to $203$ letters.
    \begin{center}
            \begin{tabular}{c c | c | c | c | c | c}
            $A$ & $B$ & $\underbrace{A}_{1 \text{ letter}}$ & $\underbrace{A\to B}_{2 \text{ letters}}$ & $\underbrace{(A\to B) \to A}_{3 \text{ letters}}$ & ... & $\underbrace{((A\to B) \to A) \to B \dots}_{203 \text{ letters}}$ \\
            \hline
            $T$ & $T$ & $T$ & & & \\
            $T$ & $F$ & $T$ & & & \\
            $F$ & $T$ & $F$ & & & \\
            $F$ & $F$ & $F$ & & & \\
            \end{tabular}
    \end{center}
    What is the total number of cells that will be $T$ among all $4$ rows and $203$ columns? (Note that the two initial $A$ and $B$ columns do \emph{not} count as columns that should be counted. However, the $\underbrace{A}_{1 \text{ letter}}$ column and all following columns do count.)
\end{qparts}
\begin{solution}
\begin{qparts}
     \item 
     Truth Table:
     \begin{center}
            \begin{tabular}{c c | c | c}
            $A$ & $B$ & $A \rightarrow B$ & $(A \rightarrow B) \rightarrow A$ \\
            $T$ & $T$ & $T$ &  $T$\\
            $T$ & $F$ & $F$ & $T$\\
            $F$ & $T$ & $T$ &  $F$\\
            $F$ & $F$ & $T$ &  $F$\\
            \end{tabular}
    \end{center}
    and so we have proved that $((A \rightarrow B) \rightarrow A) \equiv A$
    \item
    We have proved that $((A \rightarrow B) \rightarrow A) \equiv A$ \\
    Therefore $(((A \rightarrow B) \rightarrow A) \rightarrow B) \equiv (A \rightarrow B) $\\ 
    and $((((A \rightarrow B) \rightarrow A) \rightarrow B) \rightarrow A) \equiv ((A \rightarrow B) \rightarrow A)  \equiv A $. \\We can see that it returns to the original proposition and thus forms a cycle because the next letter is always $A$ and then $B$, continuing. \\
    If we replace components from the inside out to be its logical equivalence and recursively perform this operation, we can find replace all the propositions by either $A$ or $(A \rightarrow B)$. And by doing this, we can derive that propositions ended with $A$ is equivalent to $A$, and those ended with $B$ is equivalent to $(A \rightarrow B)$. \\
    And since propositions ended with $A$ have odd letters, and propositions ended with $B$ have even letters (vice versa), the 203-letter-proposition is ended with A and has the same truth value with the $A$ column. \\
    $\therefore$
    \begin{center}
            \begin{tabular}{c c | c}
            $A$ & $B$ & $\underbrace{((A\to B) \to A) \to B \dots}_{203 \text{ letters}}$ \\
            \hline
            $T$ & $T$ & $T$ \\
            $T$ & $F$ & $T$\\
            $F$ & $T$ & $F$\\
            $F$ & $F$ & $F$\\
            \end{tabular}
    \end{center}

    \item In question(b) we have justified that propositions ended with $A$ are logically equivalent to $A$ and have odd letters, while propositions ended with $B$ are logically equivalent to $A \rightarrow B$ and have even letters (vice versa), the 203-letter-proposition is ended with A and has the same truth value with the $A$ column. 
    \\$\therefore$ In the Table, odd number columns have the truth value of $TTFF$, the same as that of $A$; even number columns have the truth value of $TFTT$, the same as $(A \rightarrow B)$.
    \\ $\therefore$ There are 102 $TTFF$s and 101 $TFTT$s in the 203 columns.
    \\$\therefore$ $102\times 2 + 101 \times 3 = 505$ $T$s.

    \begin{center}
            \begin{tabular}{c c | c | c | c | c | c}
            $A$ & $B$ & $\underbrace{A}_{1 \text{ letter}}$ & $\underbrace{A\to B}_{2 \text{ letters}}$ & $\underbrace{(A\to B) \to A}_{3 \text{ letters}}$ & ... & $\underbrace{((A\to B) \to A) \to B \dots}_{203 \text{ letters}}$ \\
            \hline
            $T$ & $T$ & $T$ & $T$ & $T$& & $T$\\
            $T$ & $F$ & $T$ & $F$ & $T$ & & $T$\\
            $F$ & $T$ & $F$ & $T$ & $F$ & & $F$\\
            $F$ & $F$ & $F$ & $T$ & $F$ & & $F$\\
            \end{tabular}
    \end{center}    
\end{qparts}
\end{solution}
\subsection*{Previous Group Homework 2(2): Functionally Complete [12 points]}
A logical operator (or a set of logical operators) is considered to be \textit{functionally complete} if it can be used to make any truth table.
\begin{qparts}
    \item One set of functionally complete logical operators is $\{ \lor, \neg \}$. In other words, we can use the $\lor$ and $\neg$ operators to make any truth table. Let's test this out with an example! Consider two propositions $p$ and $q$. Write a compound proposition that is logically equivalent to $p \land q$ by only using $p$, $q$, $\lor$, $\neg$, and parentheses.
    \item Now, let's consider a new logical operator: NAND. The symbol for NAND is $\barwedge$. Below is the truth table for NAND. \emph{(If you take EECS 370, you will get to use NAND even more!)}
    \begin{center}
            \begin{tabular}{c c | c}
            $p$ & $q$ & $p \barwedge q$ \\
            \hline
            $T$ & $T$ & $F$\\
            $T$ & $F$ & $T$\\
            $F$ & $T$ & $T$\\
            $F$ & $F$ & $T$\\
            \end{tabular}
        \end{center}
        Let's start trying to figure out whether $\barwedge$ is functionally complete. Is it possible to write a proposition that is logically equivalent to $\neg p$ by only using $p$ and $\barwedge$? If so, write the proposition. If not, explain why it is impossible to do so.
    \item Is it possible to write a compound proposition that is logically equivalent to $p \lor q$ by only using $p$, $q$, $\barwedge$, and parentheses? If so, write the compound proposition. If not, explain why it is impossible to do so.
    \item Based on parts (a), (b), and (c), is $\barwedge$ functionally complete? Why or why not?
\end{qparts}

\begin{solution}
\begin{qparts}
    \item 
    $\lnot (\lnot p \lor \lnot q) \equiv (p \land q)$
    \begin{center}
        \begin{tabular}{c c | c | c |c|c| c}
        $p$ & $q$ & $p \land q$ & $\lnot p$ & $\lnot q$ & $\lnot p \lor \lnot q$ & $\lnot (\lnot p \lor \lnot q)$\\
        \hline
        $T$ & $T$ & $T$ & $F$ & $F$ & $F$ & $T$\\
        $T$ & $F$ & $F$ & $F$ & $T$ & $T$& $F$\\
        $F$ & $T$ & $F$ & $T$ & $F$ & $T$& $F$\\
        $F$ & $F$ & $F$ & $T$ & $T$ & $T$& $F$\\
        \end{tabular}
    \end{center}

    \item 
    $(p \barwedge p) \equiv \lnot p$
    \begin{center}
        \begin{tabular}{c | c | c }
        $p$ & $\lnot q$ & $p \barwedge p$ \\
        \hline
        $T$ & $F$ & $F$\\
        $F$ & $T$ & $T$\\
        \end{tabular}
    \end{center}

    \item 
    $(p \lor q) \equiv [(p \barwedge p) \barwedge (q \barwedge q)]$
    \begin{center}
        \begin{tabular}{c c| c | c | c | c}
        $p$ & $q$ & $p \lor q$ & $p \barwedge p$ & $q \barwedge q$ & $(p \barwedge p) \barwedge (q \barwedge q) $\\
        \hline
        $T$ & $T$ & $T$ & $F$ & $F$ & $T$\\
        $T$ & $F$ & $T$ & $F$ & $T$ & $T$\\
        $F$ & $T$ & $T$ & $T$ & $F$ & $T$\\
        $F$ & $F$ & $F$ & $T$ & $T$ & $F$\\
        \end{tabular}
    \end{center}

    \item Yes. We are given that \{$\lor ,\lnot$\} is functionally complete, and through (b) and (c) we have found equivalences of the two operators by using $\barwedge$. Therefore by substitution, we can say that $\barwedge$ is functionally complete.
    \end{qparts}

    \end{solution}
\end{solution}

\subsection*{\probnum Bézout's Identity [10 points]}
In number theory, there's a simple yet powerful theorem called Bézout's identity,
which states that for any two integers $a$ and $b$ (with $a$ and $b$ not both zero)
there exist two integers $r$ and $s$ such that $ar+bs=\gcd(a,b).$ Use Bézout's
identity to prove the following statements (you may assume all variables are
integers):
\begin{qparts}
    \item If $d \divides a$ and $d\divides b$, then $d \divides \gcd(a,b)$.
    \item If $a \divides bc$ and $\gcd(a,b) = 1$, then $a \divides c$.
\end{qparts}

\noindent
\textbf{Note:} $\gcd$ is short for ``greatest common divisor," so the value of $\
gcd(a,b)$ is the largest integer that evenly divides $a$ and $b.$ You won't need to
apply this definition, just know that $\gcd(a,b)$ is an integer.
\begin{solution}

\begin{qparts}
\item
Assume $d \divides a$ and $d \divides b$.\\
So there exists an int $x$ such that $a=dx$, and there exist an int $y$ such that $b = dy$.\\
Through Bézout's identity we can state that there exists two integers $r,s$ such that $ar + bs = \gcd(a,b)$.\\
Substitute $a$ and $b$:\\
\begin{align*}
    (dx)r + (dy) s = \gcd(a,b) \\
    dxr + dys = \gcd(a,b)\\
    d(xr+ys) = \gcd(a,b)
\end{align*}
Since $x,r,y,s$ are all integers, $xr+ys$ is an integer.\\
Therefore we have proved that $d \divides \gcd(a,b)$.
\item
Assume $a \divides bc$ and $\gcd(a,b) = 1$.\\
So there exists an int $x$ such that $bc = ax$.\\
Through Bézout's identity, we can state that there exists an integer $r$ and an integer $s$ such that $ar+bs = \gcd(a,b)$.\\
So:
\begin{align*}
    ar+bs = 1 \\
    car + cbs = c \\
    car + axs = c \\
    a (cr + xs) = c
\end{align*}
Since $c,r,x,s$ are integers, $cr+xs$ is an integer.\\
Therefore we proved that $a \divides c$.
\end{qparts}

\end{solution}




\subsection*{\probnum High Five! [20 points]}
Prove the following fun numerical facts:
\begin{qparts}
\item If a 5-digit integer is divisible by 4, its last two digits are also
divisible by 4. For example, 40156 is divisible by 4, and so is 56.
\item If a 5-digit integer is divisible by 3, the sum of the digits of that
integer is also divisible by 3. For example, 33762 is divisible by 3, and so is
$3+3+7+6+2=21.$
\end{qparts}
\textbf{Hint:} Think about how you can represent the digits of an integer. For
instance, if $a$ is a 2 digit number, then $a = a_1a_2 = a_1\cdot 10 +a_2\cdot 1$ (fill in the blanks).
\begin{solution}

\begin{qparts}
\item 
Let $x$ be an arbitrary 5-digit integer which can be divided by 4. \\
Since $x$ is a 5-digit integer, let $n_1,n_2,n_3,n_4,n_5$ be the  5 digits of $x$,  which means $x = 10^4 \cdot n_1 + 10 ^3 \cdot n_2 + 10^2 \cdot n_3 + 10 \cdot n_4 + n_5$ $ (n_1 \in [1,9]\; ; n_2, n_3,n_4,n_5\in [0,9]$, and all of them are integers) \\
and $10 \cdot n_4 + n_5$ is the last two digits of $x$.\\
Also, since $4 | x$, let $x = 4n$, $n$ is an integer whose value depends on $x$\\
Then we have: 
\begin{align*}
4n & = 10^4 \cdot n1 + 10 ^3 \cdot n_2 + 10^2 \cdot n_3 + 10 \cdot n_4 + n_5  \\
4n  & = 4 \cdot  2500 \cdot n_1 + 4 \cdot 250 \cdot n_2 + 4 \cdot 25 \cdot n_3 + 10 \cdot n_4 + n_5 \\
10 \cdot n_4 + n_5  &= 4 (n -  2500 \cdot n_1 + 250 \cdot n_2 + 25 \cdot n_3)
\end{align*}
Since $n1,n2,n3,n$ are integers, $(n -  2500 \cdot n_1 + 250 \cdot n_2 + 25 \cdot n_3)$ is an integer.
Therefore we prove that $4 \divides (10 \cdot n_4 + n_5)$.
  
\item 
Let $x$ be an arbitrary 5-digit integer which can be divided by 3.\\
Since $x$ is a 5-digit integer, let $n_1,n_2,n_3,n_4,n_5$ be the  5 digits of $x$,  which means $x = 10^4 \cdot n_1 + 10 ^3 \cdot n_2 + 10^2 \cdot n_3 + 10 \cdot n_4 + n_5$ $ (n_1 \in [1,9]\; ; n_2, n_3,n_4,n_5\in [0,9]$, and all of them are integers) \\
And $n_1 + n_2 + n_3 + n_4 + n_5$ is the sum of the 5 digits.\\
Also, since $3 |x$, let $x = 3n$, $n$ is an integer whose value depends on $x$. \\
Then we have:
\begin{align*}
3n & = 10^4 \cdot n1 + 10 ^3 \cdot n_2 + 10^2 \cdot n_3 + 10 \cdot n_4 + n_5  \\
3n  & = 9999 \cdot n_1 + 999 \cdot n_2 + 99 \cdot n_3 + 9 \cdot n_4 + n_1 + n_2 + n_3 + n_4 + n_5 \\
n_1 + n_2 + n_3 + n_4 + n_5 &= 
3 \cdot (3333 \cdot n_1 + 333 \cdot n_2 + 33 \cdot n_3 + 3\cdot n_4)
\end{align*}
Since $n_1,n_2,n_3,n_4$ are all integers, $(3333 \cdot n_1 + 333 \cdot n_2 + 33 \cdot n_3 + 3\cdot n_4)$ is an integer.
Therefore we prove that $3 \divides (n_1 + n_2 + n_3 + n_4 + n_5 )$.

\end{qparts}

\end{solution}
\end{document}