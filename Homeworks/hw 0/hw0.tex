\documentclass[12pt]{exam}
% essential packages
\usepackage{fullpage} % margin formatting
\usepackage{enumitem} % configure enumerate and itemize
\usepackage{amsmath, amsfonts, amssymb, mathtools} % math symbols
\usepackage{xcolor, colortbl} % colors, including in tables
\usepackage{makecell} % thicker \Xhline in table
\usepackage{graphicx} % images, resizing
% sometimes needed packages
\usepackage{wasysym} % Just used for \smiley{}
\usepackage{hyperref} % hyperlinks
\usepackage{logicproof}
\hypersetup{colorlinks=true, urlcolor=blue}
% \usepackage{logicproof} % natural deduction
% \usepackage{tikz} % drawing graphs
% \usetikzlibrary{positioning}
% \usepackage{multicol}
% \usepackage{algpseudocode} % pseudocode
% paragraph formatting
\setlength{\parskip}{6pt}
\setlength{\parindent}{0cm}
% newline after Solution:
\renewcommand{\solutiontitle}{\noindent\textbf{Solution:}\par\noindent}
% less space before itemize/enumerate
\setlist{topsep=0pt}
% creates \filcl to grey out cells for groupwork grading
\newcommand{\filcl}{\cellcolor{gray!25}}
% creates \probnum to get the problem number
\newcounter{probnumcount}
\setcounter{probnumcount}{1}
\newcommand{\probnum}{\arabic{probnumcount}. \addtocounter{probnumcount}{1}}
% use roman numerals by default
\setlist[enumerate]{label={(\roman*)}}
% creates custom list environments for grading guidelines, question parts
\newlist{guidelines}{itemize}{1}
\setlist[guidelines]{label={}, left=0pt .. \parindent, nosep}
\newlist{gwguidelines}{enumerate}{1}
\setlist[gwguidelines]{label={(\roman*)}, nosep}
\newlist{qparts}{enumerate}{2}
\setlist[qparts]{label={(\alph*)}}
\newlist{qsubparts}{enumerate}{2}
\setlist[qsubparts]{label={(\roman*)}}
\newlist{stmts}{enumerate}{1}
\setlist[stmts]{label={(\roman*)}, nosep}
\newlist{pflist}{itemize}{4}
\setlist[pflist]{label={$\bullet$}, nosep}
\newlist{enumpflist}{enumerate}{4}
\setlist[enumpflist]{label={(\arabic*)}, nosep}
\printanswers
\begin{document}
%%%%%%%%%%%%%%% TITLE PAGE %%%%%%%%%%%%%%%
\title{EECS 203: Discrete Mathematics\\
Fall 2023\\
Assignment 0: \\
Course Policies \& Page-Matching}
\date{}
\author{}
\maketitle
\vspace{-50pt}
\begin{center}
\huge Due \textbf{TUESDAY, Sept. 5}, 10:00 pm\\
\Large No late homework accepted past midnight.\\
\vspace{10pt}
\large Number of Problems: 9
\hspace{3cm}
Total Points: 100
\end{center}
\vspace{25pt}
Like all future homeworks, this assignment should be submitted as a PDF through
Gradescope (see instructions on course site for more details). We strongly prefer
students to compose homework solutions using a word processor (Google Docs or MS
Word), or ideally using \LaTeX, but we will accept handwritten homework submissions
scanned/photographed and converted to PDF. Note that submitted files must be less
than 50 mb in size, but they really should be much smaller than this. No email or
Piazza regrade requests will be accepted. For more detail on regrade requests,
please refer to course policies.
\begin{itemize}
\item We will not grade homework problems that were not properly matched on
Gradescope. Please submit early and often and double-check that you matched each
question to the page(s) where your solution appears.
\item Explanation or justification for yes/no, true/false, multiple choice
questions, and the like: You must provide some sort of explanation or justification
for these types of questions (so we know you didn’t just guess), unless explicitly
specified in the problem statement. For example, simply answering “true” for a T/F
question without providing an explanation will receive little or no credit.
\item All problems require work to be shown. Questions that require a numerical
answer will not be given credit if no work/explanation is shown.
\item Honor Code: By submitting this homework, you agree that you are in
compliance with the Engineering Honor Code and the Course Policies for 203, and
that you are submitting your own work.
\item Hyperlinks on Submitting Homework to Gradescope:
\href{https://www.gradescope.com/help#help-center-item-student-submitting}{\
textcolor{blue}{Instructions}} or
\href{https://youtu.be/KMPoby5g_nE}{\textcolor{blue}{Youtube Video}}
\end{itemize}
\newpage
% \end{foo}
%%%%%%%%%%%%%%% TITLE PAGE %%%%%%%%%%%%%%%
\subsection*{\probnum Did You Read The First Page? [6 points]}
Did you read the first page? We get it, it's a lot of words. Please give it a read!
It's important information that you'll want to know as you complete homework
assignments throughout the semester to \textbf{avoid losing points from logistical
mistakes} \smiley{}. After you've given the first page a careful read over, write
your favorite song in 2023 so far as your ``answer" to this question.
\begin{solution}
My favorite song in 2023 is \emph{Bad Habit} by \emph{Ed Sheeran}. 
\end{solution}

\subsection*{\probnum A Thing About Page Matching [6 points]}
You must match your pages to the problems on Gradescope in order to receive credit.
Please note as soon as you press submit, you’ve successfully submitted by the
deadline. You can still match the pages with no rush, that doesn’t add to your
submission time. Write down your favorite class so far to affirm you've read and
understand this.
\begin{solution}
So far, my favorite class is EECS 203 (in the first semester as a new transfer student). The course really includes many nice things including using \LaTeX.
\end{solution}

%%% Syllabus questions %%%%
\subsection*{\probnum Resource Location [12 points]}
Where can you find each of the following resources? No justification necessary.
\begin{qparts}
\item Lecture slides, lecture recordings, course announcements
\item Scheduled office hours
\item Questions and answers about course content and logistics (online forum)
\end{qparts}

\begin{solution}
\begin{qparts}
\item 'Files' and 'Lecture Recordings' in Canvas.
\item 'Office Hour Calendar'. The concrete route: Canvas/Syllabus/Office Hour Calendar
\item Piazza
\end{qparts}
\end{solution}

\subsection*{\probnum Exam dates [12 points]}
What are the dates and times of the three course exams? No justification necessary.
\begin{solution}
Exams:
\begin{enumerate}
    \item Exam 1, Wednesday, October. 4, 7:00 - 9:00 pm
    \item Exam 2, Wednesday, November. 8, 7:00 - 9:00 pm
    \item Exam 3, Thursday, December. 14, 7:00 - 9:00 pm
\end{enumerate}
\end{solution}

\subsection*{\probnum Regrade Requests [16 points]}
\begin{qparts}
\item Where do you submit a regrade request?
\item What is the deadline for submitting regrade requests? Select \textbf{one}
of the following:
\begin{qsubparts}
\item by the last day of classes
\item by the next exam
\item 24 hours after the assignment's grades have been released
\item 1 week after the assignment's grades have been released
\end{qsubparts}
\item When should you submit a regrade request? Select \textbf{any number} of
the following:
\begin{qsubparts}
\item I feel like I deserve more points
\item My solution was incorrectly graded according to the posted rubric
\item I have an alternate solution I think is right
\item I have an alternate solution I think is right, and I checked it with
an instructor in office hours or on Piazza
\end{qsubparts}
\item What 3 things should you make sure to (re)read before submitting a
regrade request?
\end{qparts}

\begin{solution}
\begin{qparts}
\item 'Regrade Requests' in Gradescope.
\item (iv)
\item (ii) (iv)
\item The posted solution, my answer, and the rubric items.
\end{qparts}

\end{solution}

\subsection*{\probnum Admin Form [8 points]}
What is the best way to contact course staff about individual administrative
concerns, and where can you find the link?
\begin{solution}
The Admin Form. We can find the link by this route: Canvas/Syllabus/Course Information/Admin Form.
\end{solution}

\subsection*{\probnum Reflection [8 points]}
What strengths, strategies, and resources do you have to help you succeed in each
of following?
\begin{qparts}
\item Weekly assignments
\item Exams
\end{qparts}
\begin{solution}
\begin{qparts}
\item For weekly assignments:
\begin{qsubparts}
\item Strength: I have some experience in algorithms which might be one of my strength.
\item Strategy: I do corresponding exercise on the textbook besides my homework to help me strengthen my understanding, and I would get inspirations from the teaching stuff and in-class studying group upon very difficult problems.
\item Resources: Piazza, ECoach, Study Groups and Office Hours.
\end{qsubparts}
\item For exams:
\begin{qsubparts}
    \item  Strength: I have positive mindset that helps me avoid being anxious.
    \item Strategy: I practice daily to train and memorize instead of merely reviewing what I learned intensively before exams.
    \item Resources: Problem Roulette for practicing; Lecture Recordings and Slides for reviewing.
\end{qsubparts}
\end{qparts}

\end{solution}

\subsection*{\probnum Exponents [16 points]}
Compute the value of each of the following. We encourage you to keep it in
exponential form as long as you can, for practice. \textbf{Remember to show your work.}
\begin{qparts}
\item $5^3$
\item $2^3\cdot 2^2$
\item $\frac{3^6}{3^3}$
\item $5^0$
\item $(2^4)^2$
\item $2^{(4^2)}$ (use a calculator for this one if you need to!)
\item $16^{\frac 12}$
\item $16^{-2}$
\end{qparts}

\begin{solution}
\begin{qparts}
\item $5^3 = 125$
\item $2^3\cdot 2^2 = 2^{3+2} = 2^5 = 32$
\item $\displaystyle \frac{3^6}{3^3} = 3^{6-3} = 3^3 = 27$
\item $5^0 = 1$
\item $(2^4)^2 = 2^8 =256 $
\item $2^{(4^2)} = 2^{16} = 2^8 \times 2^8 = 256 \times 256 = 65536$
\item $16^{\frac 12} = (4^2)^{1\over2} = 4$
\item $\displaystyle 16^{-2} = {1\over{(2^4)^2}} = {1\over{2^8}} = {1\over 256}$
\end{qparts}
\end{solution}

\subsection*{\probnum Logarithms [16 points]}
Compute the value of each of the following. If the answer does not come out
cleanly and results in an infinite decimal expansion, write ``\textit{too hard to
compute}." If the result is not defined, write ``\textit{undefined}." For any
question referencing $x$ or $y,$ let $\log_5(x)=2.6$ and $\log_5(y)=5.2.$ \
\textbf{Remember to show your work.}
\begin{qparts}
\item $\log_2{64}$
\item $\log_7{7}$
\item $\log_7{1}$
\item $\log_7{0}$
\item $\log_5(x+y)$
\item $\log_5(xy)$
\item $\log_2(x)$
\item $\log_5(x^{25})$
\end{qparts}

\begin{solution}

\begin{qparts}
\item$ \log_2{64} = \log_2{8 \times 8} = \log_2{2^3 \times 2^3} =  6$
\item $\log_7{7} = \log_7{7^1} = 1$ 
\item $\log_7{1} = \log_7{7^0} = 0$
\item $\log_7{0} = $ \emph{undefined} 
\item $\log_5(x+y) = \log_5{(5^{2.6} + 5^{5.2})} = \log_5{(5^{2.6}(1+5^{2.6}))} = 2.6 + \log_5{(1+5^{2.6})} $, can not be further computed without a calculator. Therefore the answer is too hard to compute.
\item $\log_5(xy) = \log_5{x} + \log_5{y} = 7.8$
\item $\log_2(x) = \displaystyle{{\log_5{x}}\over{\log_5{2}}} ={{2.6}\over{\log_5{2}}}$, can not be further computed without a calculator. Therefore the answer is too hard to compute.
\item $\log_5(x^{25}) = 25\log_5{x} = 25 \times 2.6 =65$
\end{qparts}

\end{solution}
\end{document}
